\usepackage[margin=1.2in,top=1.6in,bottom=1.6in]{geometry}
\usepackage[table,dvipsnames]{xcolor}
\usepackage[utf8]{inputenc}
\usepackage{svg}
\usepackage{listingsutf8}
\usepackage{graphicx}
\usepackage[makeroom]{cancel}
\usepackage{empheq}
\usepackage[most]{tcolorbox}
\usepackage{floatrow}
\usepackage[style=numeric,sorting=none]{biblatex}
% \usepackage[spanish,english]{babel}
\usepackage[rightcaption]{sidecap}
\usepackage[bottom]{footmisc}
\usepackage{listings}
\usepackage[T1]{fontenc}
\usepackage{amsmath,amssymb,amsthm}	
% \newtheorem{problem}{Problem}[chapter]
% \newtheorem{definition}{\sffamily Definition}
\usepackage{newtxtext}
\usepackage[varvw]{newtxmath}
\usepackage{esint}
\usepackage[font=small,labelfont=bf,small,sf]{caption}
\usepackage{titlesec}
\usepackage{calc}
\usepackage{fancyhdr}
\usepackage{physics}
\usepackage{hyperref}
\usepackage{lettrine}
% \usepackage{subfig}
\usepackage[cal=cm,scr=boondoxo,bb=px]{mathalfa}
\usepackage{tikz}\usetikzlibrary{shapes.misc}
\usepackage{multicol}
\usepackage{multirow}
\usepackage{booktabs}
\usepackage{subcaption}
\allowdisplaybreaks
\renewcommand{\baselinestretch}{1.2}
\addbibresource{References.bib}
%-------------------------------
%----------- COMANDOS
%-------------------------------
\newcommand{\diff}{\mathrm{d}}
\newcommand{\sol}{\normalfont\textbf{Solution}}
% \renewcommand{\dbar}{d\hspace*{-0.08em}\bar{}\hspace*{0.1em}}
\newcommand{\dem}{\normalfont\bfseries Demostración}
\newcommand{\paren}[1]{\left(#1\right)}
\newcommand{\bracket}[1]{\left[]#1\right]}
\renewcommand{\vector}[1]{\mathbf{#1}}
\newcommand{\unit}[1]{\,\hat{\mathbf{#1}}}
\renewcommand{\r}{\scalebox{0.8}{\rotatebox[origin=c]{335}{$\mathscr{n}$}}}
\newcommand{\bfr}{\scalebox{0.8}{\rotatebox[origin=c]{335}{$\mathbscr{n}$}}}
\newcommand{\ur}{\hat{\bfr}}
\renewcommand{\qedsymbol}{{\color{black!90!white}$\blacksquare$}}
%\newcommand{\eval}{\biggr\rvert}
\renewcommand{\L}{\mathcal{L}}
\newcommand{\barL}{\overline{\L}}
\renewcommand{\v}{\boldsymbol{v}}
\renewcommand{\lstlistingname}{Code}
%---------------------------------
% OPCIONES 
%---------------------------------
\setlength\parindent{20pt}
% \floatsetup[table]{capposition=beside,capbesideposition={right,bottom}}
\arrayrulecolor{black!20!white}
\setcounter{tocdepth}{2}
\setlength\fboxsep{7pt}
\setlength\fboxrule{0.8pt}
\setlength{\headheight}{17pt}
\setlength{\marginparwidth}{2in}
\floatsetup[figure]{capposition=below}
% \newtheorem{theorem}{\bfseries\sffamily Theorem}
\newcounter{theorem}
\newenvironment{theorem}{\stepcounter{theorem}\begin{tcolorbox}[colback=white,colframe=black,enhanced,breakable,sharp corners,boxrule=0.6pt]{\sffamily\textbf{Theorem \thetheorem.}}}{\end{tcolorbox}}
\newcounter{definition}
\newenvironment{definition}[1][]{\stepcounter{definition}\begin{tcolorbox}[colback=black!10!white,colframe=black,enhanced,breakable,sharp corners,boxrule=0.6pt]{\sffamily\textbf{Definition \thedefinition{\,#1}.}}}{\end{tcolorbox}}

\renewcommand{\headrulewidth}{0pt}
\let\arrow\rightarrow
\tcbset{highlight math style={enhanced,
  colframe=black,colback=white,arc=0pt,boxrule=0.5pt,sharp corners}}
\newcommand*{\figref}[2][]{%
  \hyperref[{#2}]{%
    Figure~\ref*{#2}%
    \ifx\\#1\\%
    \else
      \,#1%
    \fi
  }%
}
%---------------------------------
% ENCABEZADO
%---------------------------------
\pagestyle{fancy}
\fancyhead[CO]{\itshape\nouppercase\leftmark}
\fancyhead[RE,RO]{}
\fancyhead[LE,LO]{}
\fancyhead[CE]{\itshape\nouppercase RLC circuits}
\fancyhead[RO,LE]{\bfseries\itshape\thepage}
% \fancyheadoffset[leh,roh]{2.2in}
%---------------------------------
% ------- SECTION DESIGN
%---------------------------------
\titleformat{\section}
  {\normalfont\sffamily\Large\bfseries}{{\sffamily\bfseries\thesection.}}{1ex}{}[{\color{black!20!white}\titlerule[3pt]}]
\titleformat{\subsection}{\sffamily\large\bfseries}{\thesubsection.}{1ex}{}
\titleformat{\subsubsection}{\sffamily\bfseries}{}{0ex}{}
%---------------------------------
% ----------- EXAMPLE
%---------------------------------


%---------------------------------
%-------------- NOTE
%---------------------------------

%------------------------------
% --------- CODE
%------------------------------
\newenvironment{cd}
    {\begin{tcolorbox}[colframe=black!5!white,colback=black!5!white,enhanced,breakable,sharp corners,boxrule=0.6pt]\begin{itemize}\item[]\verbatimfont{\sffamily}}
    {\end{itemize}\end{tcolorbox}}
\newenvironment{code}
    {\verbatimfont{\sffamily}\begin{itemize}\item[]}
    {\end{itemize}}
%---------------------------------
%----- SIGNO DE SUMA CM
%---------------------------------
\DeclareSymbolFont{cmlargesymbols}{OMX}{cmex}{m}{n}
\let\sumop\relax
\DeclareMathSymbol{\sumop}{\mathop}{cmlargesymbols}{"50}
%----------------------------
% LISTING
%----------------------------
\lstdefinestyle{mystyle}{
    backgroundcolor=\color{black!10!white},   
    commentstyle=\color{black!70!white}\sffamily,
    keywordstyle=\color{black}\bfseries\sffamily,
    numberstyle=\sffamily\color{black}\footnotesize,
    stringstyle=\color{black!70!white}\itshape\bfseries\sffamily,
    basicstyle=\sffamily\footnotesize\color{black!70!white},
    breakatwhitespace=false,         
    breaklines=true,                 
    captionpos=b,                    
    keepspaces=true,                 
    numbers=left,                    
    numbersep=10pt,
    showspaces=false,                
    showstringspaces=false,
    showtabs=true,                  
    tabsize=1
}
\lstset{style=mystyle,frame=single}

\makeatletter
% the original definition in amsmath
%\def\intkern@{\mkern-6mu\mathchoice{\mkern-3mu}{}{}{}}
\def\intkern@{\mkern-8mu\mathchoice{\mkern-8mu}{}{}{}}
\makeatother

\makeatletter
\newcommand{\verbatimfont}[1]{\def\verbatim@font{#1}}%
\makeatother

\titlespacing*{\section}{0ex}{2ex}{2ex}
\titlespacing*{\subsection}{0em}{1ex}{0ex}
\titlespacing*{\subsubsection}{0ex}{1ex}{0ex}
\titlespacing*{\paragraph}{0ex}{0ex}{-1ex}
